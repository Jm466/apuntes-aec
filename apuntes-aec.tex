% -*- mode: latex; mode: flyspell; mode: auto-fill -*-

\documentclass[12pt,onecolumn]{memoir}
\usepackage[english,activeacute]{babel}
\usepackage[utf8]{inputenc}
\usepackage[hyperindex,backref,bookmarks,linktocpage]{hyperref}

\usepackage[T1]{fontenc}

\usepackage{textcomp}
\usepackage[sc]{mathpazo}
\usepackage{helvet}
\linespread{1.05} % Palatino needs more leading (space between lines)

\usepackage{enumerate}
\usepackage{epsfig}
\usepackage{subfig}
\usepackage{amsmath}
\usepackage{multirow}
\usepackage{color}
\usepackage{setspace}
\usepackage{rotating}
\usepackage{colortbl}

\hypersetup{ 
  pdfauthor = {Alberto Ros Bardisa}, 
  pdftitle = {Apuntes de Ampliación de Estructuras de Computadores}, 
  pdfkeywords = {arquitectura de computadores, rendimiento, segmentación, memoria}, 
}

\DeclareTextFontCommand{\c}{\sffamily \slshape}

\setstocksize{297mm}{210mm}
\settrimmedsize{297mm}{210mm}{*}
\setlength{\trimtop}{0pt}
\setlength{\trimedge}{\stockwidth}
\addtolength{\trimedge}{-\paperwidth}
\setlrmarginsandblock{35mm}{25mm}{*}
\setulmarginsandblock{45mm}{50mm}{*}
\setheadfoot{\onelineskip}{3\onelineskip}
\setheaderspaces{*}{1.2\onelineskip}{*}
\checkandfixthelayout

%\parskip 	10pt

\pagestyle{ruled}

\setsecnumdepth{all}

\renewcommand{\topfraction}{1}
\renewcommand{\bottomfraction}{1}
\renewcommand{\textfraction}{0}
\renewcommand{\floatpagefraction}{0}

\makechapterstyle{tesis}{
  \setlength{\beforechapskip}{70pt}
  \renewcommand*{\chapnamefont}{%
    \normalfont\Huge\scshape\raggedleft}
  \renewcommand*{\chaptitlefont}{%
    \normalfont\Huge\bfseries\sffamily\raggedleft}
  \renewcommand*{\afterchapternum}{%
    \par\hspace{1.5cm}\hrule\vskip\midchapskip}
}

\chapterstyle{tesis}

\setsecheadstyle{\Large\bfseries\sffamily\raggedright}
\setsubsecheadstyle{\large\bfseries\sffamily\raggedright}
\setsubsubsecheadstyle{\normalsize\bfseries\sffamily\raggedright}
\setparaheadstyle{\normalsize\bfseries\sffamily}
\setsubparaheadstyle{\normalsize\bfseries\sffamily}

\begin{document}

\hyphenation{ge-ne-ra-dos}

\sloppy
\widowpenalty=10000
\clubpenalty=10000

\title{Apuntes de Ampliación de Estructuras de Computadores}

\author{Alberto Ros Bardisa}

% Begin of title page
\thispagestyle{empty}

\begin{center}

\begin{figure}
\centering
%  \epsfig{file=eps/escudo_universidad,width=6cm}
%  \epsfig{file=eps/escudo.ps,width=6cm,viewport=0 -40 690 639}

  \Large

  %\vspace{1cm}

  \textsc{Universidad de Murcia}

  \large

  Departamento de Ingeniería y \\
  Tecnología de Computadores

  \vspace{3cm}

\end{figure}

\huge

\textbf{Apuntes de\\ Ampliación de Estructura de Computadores}

\vspace{7cm}

\normalsize

Alberto Ros Bardisa

\vspace{0.5cm}

Murcia, septiembre de 2020

\end{center}
% End of title page

\cleardoublepage 

\chapter*{Resumen}
\addcontentsline{toc}{chapter}{Resumen}

Estos apuntes pretenden servir como guía para la asignatura
``Ampliación de Estructura de Computadores'' del Grado en Ingeniería 
Informática y Programación Conjunta de Estudios Oficiales Grado
en Matemáticas y Grado en Ingeniería Informática.


\setcounter{chapter}{-1}

\def\figurename{Figura}
\def\tablename{Tabla}
\def\chaptername{Tema}

\chapter{Presentación de la asignatura}
\label{cap:presentacion}

Esta asignatura pertenece al área de ``Arquitectura y Tecnología de
Computadores'' y es la tercera de una serie de asignaturas del grado
destinadas a comprender cómo funcionan los computadores y cómo
hacerlos más eficientes.

En ``Fundamentos de Computadores'' se estudia la codificación de la
información y los sistemas digitales combinacionales. En ``Estructura
y Tecnología de Computadores'' se estudian sistemas digitales
secuenciales, un procesador básico y la jerarquía de memoria. En
``Ampliación de Estructura de Computadores'' se hace énfasis en el
rendimiento del computador, se estudia un procesador más avanzado
(segmentado) y aspectos avanzados de la jerarquía de
memoria. Finalmente, en ``Arquitectura y Organización de
Computadores'' se estudian aspectos como el diseño de un procesador
superescalar, la sincronización, el protocolo de coherencia de cachés
y los modelos de consistencia de memoria.

En particular, en ``Ampliación de Estructura de Computadores'' vamos a
ver cómo se pueden mejorar las prestaciones de los computadores por
primera vez en el grado. Hasta el momento nos hemos centrado en tener
un computador que funcione correctamente, sin mirar mucho su
rendimiento. En este curso vamos a construir un ordenador mas potente,
gracias a mejoras ``arquitectónicas''. Se puede decir, por tanto, que
en esta asignatura comienza lo que se conoce por \emph{arquitectura de
  computadores}. No solo vamos a aprender a identificar qué diseño es
mejor que otro de forma ``cualitativa'' sino que también vamos a
aprender a decir de forma ``cuantitativa'' cuantas veces mejor es un
diseño que otro.

Nota: La asignatura tiene asignados 6 ECTS. Es decir, se requieren
unas 150 horas de trabajo por parte del alumno: 60 horas presenciales
y 90 horas no presenciales. Esto significa que por cada hora
presencial, el alumno debe dedicar 1,5 horas más al estudio de la
asignatura.

\section{Introducción}
\label{sec:introduccion_presentacion}

Ley de Moore: Los continuos avances en la escala de integración
permiten reducir el tamaño de los transistores y por tanto el número
de transistores que hay dentro de un chip. El número de transistores
por chip se dobla cada año y medio o dos años.

Las mejoras en el rendimiento de los computadores desde su inicio han
sido debidas tradicionalmente a dos motivos principales: mejoras
debidas a la tecnología y mejoras debidas a la arquitectura. Hasta los
70, las mejoras de ambos eran similares. A partir de los 70 las
mejoras en la tecnología han dominado.



%%%%%%%%%%%%%%%%%%%%%%%%%%%%%%%%%%%%%%%%%%%%%%%%%%%%%%%%%%%%%%%%%%%%%%%%%%%%%%%%%%%%%%%%%%%%

\chapter{Análisis de prestaciones en arquitectura de computadores}
\label{cap:prestaciones}


\section{Introducción}
\label{sec:introduccion_prestaciones}

¿Podríamos medir el rendimiento o prestaciones de un computador con un
solo número? En realidad para ser precisos necesitaríamos más de uno,
pero por la necesidad de obtener de forma rápida si un computador
ofrece mejores prestaciones que otro, tratamos de definir las
prestaciones con un solo número. Veremos en este tema que no es una
tarea fácil.

A la hora de usar una métrica para medir el rendimiento debemos llevar
mucho cuidado y saber interpretarla, ya que puede dar lugar a
confusión.

Pongamos un ejemplo. ¿Cómo mediríamos las prestaciones de un coche?
¿Velocidad? Lo importante en realidad no es qué velocidad pueda
alcanzar, sino cuanto tiempo va a tardar en ir de un origen a un destino. Un
coche más rápido que toma una ruta más larga puede tardar más que uno
más lento. Además, no es lo mismo si tenemos un coche con una
capacidad de 5 pasajeros o un autobús con una capacidad de 50. El
autobús, aunque más lento, muy probablemente realice el trabajo de
llevar 100 pasajeros de una ciudad a otra antes que el coche.
Lo que nos interesa en realidad es el trabajo útil que hace el coche
por unidad de tiempo. La clave está en el tiempo.

\section{Definición de rendimiento}

Podemos definir el rendimiento como la cantidad de trabajo realizado
por el computador por unidad de tiempo.

Si estamos interesados en comparar dos computadores respecto al mismo
trabajo, entonces podemos definir rendimiento como la inversa del
tiempo de ejecución.

\[ Rendimiento_{x} = \frac{1}{Tiempo\_de\_ejecución_{x}} \]

Para comparar el rendimiento de dos computadores diremos que $x$ es un
n\% más rápido que $y$ si:

\[ \frac{Rendimiento_{x}}{Rendimiento_{y}} = 1 + \frac{n}{100} \]

Por tanto:

\[ \frac{Rendimiento_{x}}{Rendimiento_{y}} =
\frac{\frac{1}{Tiempo\_de\_ejecución_{x}}}{\frac{1}{Tiempo\_de\_ejecución_{y}}}
= \frac{Tiempo\_de\_ejecución_{y}}{Tiempo\_de\_ejecución_{x}} = 1 + \frac{n}{100}\]


\section{Tiempo de ejecución}

Fórmula del tiempo de ejecución \[ T_{ej} = NI \times CPI \times
T_{c} \]

Fórmula del CPI \[ CPI = \sum_{i=1}^{n} CPI_i \times f_i \]

\subsection{Ejercicio}

Tenemos la siguiente distribución de instrucciones:

\begin{center}
\begin{tabular}{lrr}
\hline
{\textbf{Tipo}} & {\textbf{Porcentaje}} & {\textbf{Ciclos}} \\
\hline
ALU & 33\% & 2 \\
Load & 21\% & 3 \\
Store & 12\% & 3 \\
Branch & 24\% & 3 \\
FP & 10\% & 12 \\
\hline
\end{tabular}
\end{center}

Sabemos que podemos reducir el número de ciclos de las operaciones
de FP a la mitad, pero ello nos provocará un incremento del ciclo
de reloj del 25\%. Además ahora las cargas consumirán un ciclo
más.

\begin{enumerate}

\item[a)] ¿Cuál de las máquinas es más rápida ahora? ¿En qué factor?.
\item[b)] Si conseguimos no afectar a las cargas con la modificación, ¿varía
el resultado obtenido? ¿Ahora qué máquina es la más rápida? ¿En
qué factor?.

\end{enumerate}

\subsection{Solución}

Se trata de calcular en ambos casos el tiempo de CPU usando la fórmula explicada en la teoría:
\[ T_{CPU} = NI \times CPI \times T_{ciclo} \]

A partir de los resultados se determina qué máquina es más rápida y se calcula el factor de mejora.

\subsection*{Apartado a}

Empezamos por la configuración original (sin aplicar la optimización que describe el enunciado):
\[ T_{CPU\_orig} = NI_{orig} \times CPI_{orig} \times T_{ciclo\_orig} \]

donde $CPI_{orig} = 0.33 \times 2 + 0.21 \times 3 + 0.12 \times 3 + 0.24 \times 3 + 0.10 \times 12 = 3.57$, así pues nos queda que:
\[ T_{CPU\_orig} = NI_{orig} \times 3.57 \times T_{ciclo\_orig} \]

Para la configuración alternativa el tiempo de ciclo se incrementa un 25\% con respecto al original, y los ciclos que tardan las operaciones FP y las cargas se modifican. Tenemos que:
\begin{align*}
  T_{ciclo\_alt} &= 1.25 \times T_{ciclo\_orig}\\
  CPI_{alt} &= 0.33 \times 2 + 0.21 \times 4 + 0.12 \times 3 + 0.24 \times 3 + 0.10 \times 6 = 3.18
\end{align*}

El número de instrucciones no cambia ($NI_{alt} = NI_{orig}$), por lo que:
\[ T_{CPU\_alt} = NI_{alt} \times 3.18 \times T_{ciclo\_alt} = NI_{orig} \times 3.18 \times 1.25 \times T_{ciclo\_orig} = NI_{orig} \times 3.975 \times T_{ciclo\_orig} \]

Por lo tanto, la configuración original es $1.1134$ veces ($3.975 / 3.57$) más rápida que la alternativa propuesta, o lo que es lo mismo: un 11.34\%.

\subsection*{Apartado b}

En este caso hay que volver a calcular el CPI de la configuración alternativa teniendo en cuenta que el número de ciclos de las cargas no se ve afectado:
\begin{align*}
CPI_{alt} &= 0.33 \times 2 + 0.21 \times 3 + 0.12 \times 3 + 0.24 \times 3 + 0.10 \times 6 = 2.97 \\
T_{CPU\_alt} &= NI_{orig} \times 2.97 \times 1.25 \times T_{ciclo\_orig} = NI_{orig} \times 3.7125 \times T_{ciclo\_orig}
\end{align*}

La configuración original sigue siendo más rápida, esta vez en un factor 1.04 ($3.7125/3.57$).

\section{Métricas populares}

Otras métricas como MIPS (millones de instrucciones por segundo) y
MFLOPS (millones de operaciones de punto flotante por segundo) usado
en \url{top500.org} son métricas de velocidad y tienen una serie de
desventajas.

Volviendo al símil del coche, los kilómetros recorido sería el
equivalente al número de instrucciones ejecutadas por el programa.

Un computador puede ofecer más MIPS a costa de requerir ejecutar más
instrucciones para realizar la misma tarea que otro programa con menos
instrucciones.

\section{La Ley de Amdahl}

La ley de \href{https://es.wikipedia.org/wiki/Gene_Amdahl}{Amdahl} es una 
fórmula matemática que describe la mejora total en el rendimiento de un 
sistema(procesador en nuestro caso) con respecto a la mejora de una fracción 
de dicho sistema, es decir, si mejoramos una parte CONOCIDA de una tarea un 
porcentaje CONOCIDO, esta fórmula nos permite saber el porcentaje que ha mejorado
todo el procesador PARA ESOS VALORES.

Por ejemplo: En un procesador vamos a mejorar las instrucciones en coma flotante para
que vayan el doble de rápidas, pero solamente ocupan el 10\% del tiempo de ejecución 
con nuestro programa; en este caso podemos aplicar la Ley de Amdahl porque sabemos que
parte estamos mejorando(el 10\% del total) y sabemos que porcentaje estamos mejorando
(el doble, o sea un 100\%).
(Este ejercicio está resuelto en las transparencias, página 19) %Alguien debería de escribirlo en estos apuntes :/
Cabe destacar que que la mejora no es global, pues al usar otro programa, seguramente
cambie cuanto porcentaje del tiempo se dedica a las instrucciones de coma flontante,
por tanto la mejora total no sería la misma que en el caso anterior.

\section{Cómo comparar resultados}

\section{Programas de prueba (benchmarks)}

%%%%%%%%%%%%%%%%%%%%%%%%%%%%%%%%%%%%%%%%%%%%%%%%%%%%%%%%%%%%%%%%%%%%%%%%%%%%%%%%%%%%%%%%%%%%

\chapter{Segmentación básica}
\label{cap:segmentado}


\section{Introducción}
\label{sec:introduccion_segmentado}

Concepto de segmentación. Segmentación para la ejecución de
instrucciones: camino de datos y control segmentado. Riesgos de la
segmentación (estructurales, de datos y de control). Excepciones y su
implementación.

\subsection{El juego de instrucciones DLX}

Repaso de instrucciones. Remarcar diferencias con MIPS (ETC): blt -> slt \& beqz

\subsection{Procesador multiciclo para DLX}

Repaso de etapas multiciclo (ETC), pero adaptado a DLX. Ejemplo de
un load, que es el más largo (5 ciclos). El resto de instrucciones se adaptan a éste. 

\section{Segmentación para la ejecución de instrucciones}

Describir segmentación a grandes rasgos, o de forma esquemática. Los
detalles de implementación se verán al final.

Ejercicio cuantitativo de segmentación.

\section{Problemas de la segmentación: los riesgos}

Si tuvieramos una fábrica de coches la segmentación ya funcionaría
bien y tendríamos un cauce ideal, produciendo un coche por tiempo de
cada etapa.

Pero las instrucciones son más complicadas. No todas son iguales. Y la
ejecución de una puede depender de la otra.

Hay 3 tipos de riesgos: estructurales, de datos y de control. 

\subsection{Riesgos estructurales}

Opciones: duplicar, parar, evitar sw (separar o nops).

\subsection{Riesgos de datos}

Opciones: parar, forwarding, evitar sw (separar o nops).

\subsection{Riesgos de control}

Opciones: parar, adelantar calculo salto, predecir, evitar sw (delay-slot o nops).

Detalle de la implementación. ¿Cómo parar ante riesgos (y adelantar el
calculo del salto)?

Ejercicios de delay en calculo de tiempo.

%%%%%%%%%%%%%%%%%%%%%%%%%%%%%%%%%%%%%%%%%%%%%%%%%%%%%%%%%%%%%%%%%%%%%%%%%%%%%%%%%%%%%%%%%%%%

\chapter{Segmentación avanzada y predicción de saltos}
\label{cap:avanzada}


\section{Introducción}
\label{sec:introduccion_avanzada}

Mitigación de los riesgos de datos: adelantamientos. Mitigación de los riesgos de control (predicción de saltos estática y predicción de saltos dinámica). Segmentación DLX de punto flotante. Cauces con terminación fuera de orden.

\section{Predicción de saltos}

Observación 1: los saltos tienden a tener el mismo comportamiento.

Observación 2: saltos correlacionados.

\subsection{Estática}

Algo más sobre predicción estática saltos para alante o para atrás o comparación == 0 o != 0.

\subsection{Contador saturado por salto}

Ejemplo: bucle.

Aliasing.

\subsection{Correlación}

ejemplo: if d == 0
ejemplo: found = find(elem); if (found) ...  if a[i] = elem.

historia global.

gshare.

%%%%%%%%%%%%%%%%%%%%%%%%%%%%%%%%%%%%%%%%%%%%%%%%%%%%%%%%%%%%%%%%%%%%%%%%%%%%%%%%%%%%%%%%%%%%

\chapter{Sistema de memoria de altas prestaciones}
\label{cap:memoria}


\section{Introducción}
\label{sec:introduccion_memoria}

Introducción a los sistemas de memoria. Diseño de una jerarquía de caches de alto rendimiento: reducción de la tasa de fallos,  reducción de la penalización por fallo y reducción del tiempo de acierto en caches. Organizaciones de la memoria principal. Memoria virtual. Técnicas de traducción rápida de direcciones virtuales. Protección de memoria.

\section{Evaluación del Rendimiento de la Jerarquía de Memoria}

\section{Reducción de la Tasa de Fallos de Caché}

\subsection{Clasificación de fallos de caché}

Podemos distinguir los siguientes motivos por los que se 
producen fallos en caché:

Forzosos: el primer acceso a un bloque de memoria no puede 
estar en la caché (poco efecto en programas grandes)
- Llamados fallos de arranque en frío o de primera referencia

Capacidad: la memoria caché no tiene el tamaño suficiente para 
contener todos los bloques necesarios
- En un momento determinado uno de los bloques que se han 
utilizado tiene que dejar hueco a otro (la caché está llena)
- Se produce fallo si se vuelve a referenciar el bloque desalojado

Conflicto: la memoria caché no es totalmente asociativa 
- Aciertos en una caché totalmente asociativa que se vuelven 
fallos en una asociativa por conjuntos de 
n-vías se deben a más de n peticiones sobre algunos conjuntos

\subsection{Optimizaciones hardware}

\subsubsection{Aumento del tamaño de bloque}

Y del tamaño de la caché. De hecho la primera figura (pag 29) lo contempla.

\subsubsection{Aumento de la asociatividad}

\subsubsection{Caché de víctimas}

\subsubsection{Búsqueda anticipada de datos e instrucciones (prefetching)}

\subsection{Optimizaciones del compilador}

\subsubsection{Combinación de arrays (merging arrays)}

\subsubsection{Intercambio de iteraciones (loop exchange)}

\subsubsection{Unión de bucles (loop fusion)}

\subsubsection{Blocking}

\subsubsection{Búsqueda anticipada}


\section{Reducción de la Penalización por Fallo de Caché}

\section{Reducción del Tiempo en Caso de Acierto en Caché}

\section{Organizaciones de la Memoria Principal}

\bibliographystyle{plain} 
\bibliography{bibliografia} 

\end{document} 
